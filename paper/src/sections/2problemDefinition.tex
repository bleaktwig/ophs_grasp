Let G be a complete $G=(V,E)$ where $V$ is the union of the subsets $M$ and $N$. $M$ itself is a set of $H$ hotels $h=0,..,H-1$, while $N$ is a set of $P$ vertices $p=H,...,H+P-1$, named Points of Interest (POIs) by Divsalar et al.~\cite{divsalar2014} The travel times between each pair of vertices $i,j$ is usually considered symmetric and is given by $b_{ij}$. These travel times can be based on Euclidean Distances or given by a travel time matrix.

For clarification, an ordered set of POIs starting and ending in a hotel is referred to as a trip, and the length of each trip $t$ is limited by a time budget $B_t$. The ordered set of trips with a starting and ending hotel is called a tour. The objective of the problem is to maximize the score $S_p$ collected by visiting different POI $p$, while each must be visited at most once without violating the time budget constraint.

The most challenging part of the OPHS is the Hotel Selection, because choosing a different hotel significantly affects the whole solution. It is also worth noting that, according to Divsalar et al.~\cite{divsalar2014}, it isn't possible to predict which selection of hotels will result in a good solution, and that all variants of the OP are known for how the near-optimal solutions for them are usually separated and possibly even far apart~\cite{divsalar2013}.

A description of three associated problems is now given, to understand the difference between each and the OPHS and also to later observe algorithms used to tackle each one of them:

\textbf{Orienteering Problem (OP)}: A simpler graph $G=(N,E)$ is given, where $N$ consists only of $P$ POIs. The objective this time is to maximize the total collected score in a trip, given that the time budget is not exceeded. It is worth noting that the OP can be viewed as a combination between two classical combinatorial problems, the Knapsack Problem and the Travelling Salesman Problem (TSP)~\cite{vansteenwegen2011}. Comprehensive surveys about the OP and its variants can be found in the works of Feillet et al.~\cite{feillet2005}, Laporte and Rodr\'iguez-Mart\'in~\cite{laporte2007}, Vansteewegen et al.~\cite{vansteenwegen2011}, Archetti et al.~\cite{archetti2014} and Gunawan et al.~\cite{gunawan2016}.

\textbf{Team Orienteering Problem (TOP)}: Introduced originally by M. Chao in 1996~\cite{chao1996}, this problem extends the OP by determining $m$ trips instead of just one, thus making this a more related problem to the OPHS than the original OP since it deals with various trips.

\textbf{Travelling Salesperson Problem with Hotel Selection (TSPHS)}: Presented by Vansteenwegen et al.~\cite{vansteenwegen2012}, it extends the TSP with the need for stops in-between the path for the salesperson to rest. Its difference with the OPHS is that, since it's based on the TSP, it requires the first and the last hotels visited to be the same and instead of assigning a score to each customer it tries to minimize the time in which all customers are visited.

It is important to note that each sub-problem of selecting a trip between two hotels can be considered an OP or a TOP by itself, therefore it is useful to study both problems by themselves to apply solution-finding algorithms for each in the optimization of each trip.

Something that might be obvious but still worth noting is that the Total Number of Feasible Sequences of Hotels (TNFS) is crucial in predicting the computational effort required to find the near-optimal or optimal solutions, but, as of today, no publication has found an efficient way to calculate this number. Divsalar et al. talks about this with more detail in~\cite{divsalar2014}.

Four sets of benchmark instances were designed by Divsalar et al. in~\cite{divsalar2013} to test the performance of OPHS solving algorithms. Apart from providing these instances, the author also explains a simple method to develop instances for the OPHS based on instances for the OP which retain the global optima for each. Instances for the OP and for the OPHS can be found at \textless www.mech.kuleuven.be/cib/op\textgreater.